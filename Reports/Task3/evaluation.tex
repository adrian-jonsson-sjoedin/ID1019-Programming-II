\documentclass[a4paper,11pt]{article}
\usepackage{graphicx}
\usepackage[utf8]{inputenc}
\usepackage{hyperref}
\usepackage{placeins}
\usepackage[newfloat]{minted}
\usepackage{caption}
\usepackage{amssymb}

\newenvironment{code}{\captionsetup{type=listing}}{}
\SetupFloatingEnvironment{listing}{name=Code Overview}


\hypersetup{
    colorlinks=true,
    linkcolor=blue,
    filecolor=black,      
    urlcolor=blue,
    citecolor=black,
}

\begin{document}

\title{
    \textbf{Task 3 Evaluation of an Expression}
}
\author{Adrian Jonsson Sjödin}
\date{Spring Term 2023}

\maketitle

\section*{Introduction}
The purpose of this task was to implement a program that evaluates a mathematical expression containing variables, and returns the result 
given an environment for the variables. 

\section*{Method}
The way to solve this task was quite similar to the earlier derivatives task. Like there we started with defining the different allowed 
expressions and what form a literal will take. 

After this we started with implementing the functions given in the skeleton code, these being the {\tt eval/2} functions. Here, like in the previous tasks, we start with
considering the base cases of what should happen when we want to evaluate a literal. For the number we simply return the number, while for the variable we 
use Elixir's {\tt Map} module to retrieve the value bound to the variable. For the quotient we have to do a bit more. We want to evaluate fractions to the lowest common 
denominator, as well as not have zero as a divisor. To do this we use a conditional {\tt if else} clause and the {\tt Integer} module's {\tt gcd/2} function. The code 
for this can be seen in code overview \ref{code:evalQuot}.

Having done the base cases for evaluation of a literal we now need to implement how to evaluate an expression. This is done by recursively calling the {\tt eval/2} function
until it returns the value for the number or variable. When we have that it will go back through the functions and call the corresponding arithmetic functions. This way we will
recursively calculate the expression. Furthermore if any of the expressions return undefined during this process, for example because of division by zero, we want the function 
to immediately return {\tt :undefined} without making any further calculations. To achieve this we add a conditional check before each recursive call to {\tt eval/2}. This can 
be seen in code overview \ref{code:evalAdd} where we have the code for when we evaluate an addition between two expressions.

\begin{code}
\captionof{listing}{Evaluation of quotient}
\label{code:evalQuot}
\begin{minted}{elixir}
def eval({:quotient, dividend, divisor}, _environment) do
    if divisor == 0 do :undefined
    else
        gcd = Integer.gcd(dividend, divisor)
        {:quotient, dividend/gcd, divisor/gcd}
    end
end
\end{minted}
\end{code}
\begin{code}
\captionof{listing}{Evaluation of addition between two expressions}
\label{code:evalAdd}
\begin{minted}{elixir}
def eval({:add, expression1, expression2}, environment) do
    left = eval(expression1, environment)
    right = eval(expression2, environment)
    if left == :undefined || right == :undefined do
        :undefined
    else
    add(left, right)
    end
end
\end{minted}
\end{code}

The arithmetic functions {\tt add/2, subtract/2, multiply/2} and\\ {\tt divide/2} are private functions that takes the number passed along to them and 
is what does the calculation of the expressions. When writing them it is important that we define them in the correct order and that we have a function
for each possible combination that can exist. We must start with operations between two quotients, then number and quotient, and lastly between two numbers. 
If we have between two numbers first the program will use that to try and calculate everything and it will crash. 

Lastly we implemented a {\tt pretty\_print/1} function, like in earlier tasks, for easier testing and evaluation of expressions. The code for this is really similar to 
earlier tasks so we will omit it here.


\section*{Result}
In code overview  we can see the result from a few test and conclude that the program works as intended.
\begin{code}
\captionof{listing}{Caption here}
\label{code:classStructure}
\begin{minted}{bash}
    iex(5)> Evaluation.test1  
    Expression: 2*x + 3 + 1/2
    The environment is: %{x: 2}
    Expected: 15/2
    Result, pretty format: 15.0/2.0
    Result: {:quotient, 15.0, 2.0}
    iex(6)> Evaluation.test2
    Expression: (2*a + c + 6/8) / (4)
    The environment is: %{a: 1, b: 2, c: 3, d: 4}
    Expected: 23/16
    Result, pretty format: 23.0/16.0
    Result: {:quotient, 23.0, 16.0}
    iex(7)> Evaluation.test3
    Expression: (2) / (0) + x + y
    The environment is: %{x: 2, y: 3}
    Expected: :undefined
    Result, pretty format: Error: invalid expression
    Result: :undefined
\end{minted}
\end{code}


\section*{Discussion}
This task was quite difficult to get started with initially since I did not understand exactly what we where supposed to do at first and how exactly this was 
suppose to connect with the video lectures for week 5. However after asking a classmate some questions and understanding what to do it ended up being quite 
straightforward to program. The hardest part of the task was to make sure we returned {\tt :undefined} and stopped evaluating expressions if one of the arguments 
was zero. But after some research into it I managed to make it work and the end result is as seen in code overview \ref{code:evalAdd}.

For those interested the code can be found here: 
\href{https://github.com/adrian-jonsson-sjoedin/ID1019-Programming-II/tree/main/Task3_Solution}{GitHub}.




\end{document}
