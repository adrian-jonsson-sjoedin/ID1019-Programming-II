\documentclass[a4paper,11pt]{article}
\usepackage{graphicx}
\usepackage[utf8]{inputenc}
\usepackage{hyperref}
\usepackage{placeins}
\usepackage[newfloat]{minted}
\usepackage{caption}
\usepackage{amssymb}

\newenvironment{code}{\captionsetup{type=listing}}{}
\SetupFloatingEnvironment{listing}{name=Code Overview}


\hypersetup{
    colorlinks=true,
    linkcolor=blue,
    filecolor=black,      
    urlcolor=blue,
    citecolor=black,
}

\begin{document}

\title{
    \textbf{Task 3 Evaluation of an Expression}
}
\author{Adrian Jonsson Sjödin}
\date{Spring Term 2023}

\maketitle

\section*{Introduction}
The purpose of this task was to implement a program that evaluates a mathematical expression containing variables, and returns the result 
given an environment for the variables. 

\section*{Method}
The way to solve this task was quite similar to the earlier derivatives task. Like there we started with defining the different allowed 
expressions and what form a literal will take. 

After this we started with implementing the functions given in the skeleton code, these being the {\tt eval/2} functions. Here, like in the previous tasks, we start with
considering the base cases of what should happen when we want to evaluate a literal. For the number we simply return the number, while for the variable we 
use Elixir's {\tt Map} module to retrieve the value bound to the variable. For the quotient we have to do a bit more. We want to evaluate fractions to the lowest common 
denominator, as well as not have zero as a divisor. To do this we use a conditional {\tt if else} clause and the {\tt Integer} module's {\tt gcd/2} function. The code 
for this can be seen in code overview \ref{code:evalQuot}

Having done the two base cases for when 

\begin{code}
\captionof{listing}{Evaluation of quotient}
\label{code:evalQuot}
\begin{minted}{elixir}
def eval({:quotient, dividend, divisor}, _environment) do
    if divisor == 0 do :undefined
    else
        gcd = Integer.gcd(dividend, divisor)
        {:quotient, dividend/gcd, divisor/gcd}
    end
end
\end{minted}
\end{code}

Code
\begin{minted}{elixir}
    def deriv({:exp, u, {:num, n}}, v) do
    {:mul, 
        {:mul, {:num, n}, {:exp, u, {:num, n - 1}}},
        deriv(u, v)
    }
  end
\end{minted}

Code in text {\tt p\_print/1} and {\tt p\_print/2}.

\section*{Result}
Code with caption and label for referencing
\begin{code}
\captionof{listing}{Caption here}
\label{code:classStructure}
\begin{minted}{elixir}
iex(13)> Deriv.test_all()  
Expression: ((((((cos(x) + (5*x)) + ((x)^(3) + 4)) + sin(x)) + 
sqrt(x)) + ln((x)^(2))) + (4/(((x*3) + 2))^(2)))

Derivative of expression: ((((((((-1*1)*sin(x)) + ((0*x) + 
(5*1))) + (((3*(x)^(2))*1) + 0)) + (1*cos(x))) + (1/(2*sqrt(x))))
 + (((2*(x)^(1))*1)/(x)^(2))) + (((4*-2)*(((1*3) + (x*0)) + 0))/
(((x*3) + 2))^(3)))

Simplified: (((((((-1*sin(x)) + 5) + (3*(x)^(2))) + cos(x)) + 
(1/(2*sqrt(x)))) + ((2*x)/(x)^(2))) + (-24/(((x*3) + 2))^(3)))
:ok
\end{minted}
\end{code}


\section*{Discussion}
Link
\href{https://github.com/adrian-jonsson-sjoedin/ID1019-Programming-II/tree/main/Task1_Solution}{GitHub}.




\end{document}
