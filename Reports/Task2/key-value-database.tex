\documentclass[a4paper,11pt]{article}
\usepackage{graphicx}
\usepackage[utf8]{inputenc}
\usepackage{hyperref}
\usepackage{placeins}
\usepackage[newfloat]{minted}
\usepackage{caption}
\usepackage{amssymb}

\newenvironment{code}{\captionsetup{type=listing}}{}
\SetupFloatingEnvironment{listing}{name=Code Overview}


\hypersetup{
    colorlinks=true,
    linkcolor=blue,
    filecolor=black,      
    urlcolor=blue,
    citecolor=black,
}

\begin{document}

\title{
    \textbf{Task 2 Key Value Database}
}
\author{Adrian Jonsson Sjödin}
\date{Spring Term 2023}

\maketitle

\section*{Introduction}
In this task we implement two different versions of a key-value database, that is used to look up the value associated with a key. The keys can be 
anything (atom, int, float, etc.), since we simply sort them using the regular "{\tt <}" operator. Just keep in mind that in elixir the following apply:\\
{\tt number < atom < reference < function < port < pid < tuple < map < list < bitstring}


\section*{Method}
We started with implementing the map using a list since that seemed the easiest. To get started we first watched the prerecorded videos on recursion 
and trees. Especially the videos about recursion proved helpful since the addressed how one handles lists, and program recursively.

Having watched those we started with the {\tt add/3} function, and the trick here was to cover all the base cases. For {\tt add/3} being 
1) add to an empty map, 2) the key we want to add already exists and 3) add a new pair to an empty map.

The code for {\tt add/3} is surprisingly simple and can be seen in code overview \ref{code:listAdd}. 

\begin{code}
\captionof{listing}{{\tt add/3}}
\label{code:listAdd}
\begin{minted}{elixir}
def add([], key, value), do: [{key, value}]
def add([{key, _} | map], key, value), do: [{key, value} | map]
def add([head | map], key, value), do: [head | add(map, key, value)]
\end{minted}
\end{code}
The first {\tt add/3} covers the case when we have an empty map, the second the case when we want to change the value, and then we have the third
one which will recursively call the {\tt add/3} function and do either the first or second {\tt add/3} on the tail that is passed along, or the third
{\tt add/3} again.

Both the {\tt lookup/2} and {\tt remove/2} functions work similarly. We first identify all the cases and then write the corresponding functions matching
those cases.
For {\tt lookup/2} we have when the map is empty, when we've found the key, and when we still have'nt found it, and for {\tt remove/2} we have the same 
cases.

The code for this can be seen in code overview \ref{code:listLookRemove}
\begin{code}
\captionof{listing}{{\tt lookup/2} and {\tt remove/2}}
\label{code:listLookRemove}
\begin{minted}{elixir}
def lookup([], _key), do: nil
def lookup([{key, _value}=pair | _], key), do: pair
def lookup([_ | map], key), do: lookup(map, key)
def remove([], _), do: nil
def remove([{key, _} | map], key), do: map
def remove([head|map], key), do: [head | remove(map, key)]
\end{minted}
\end{code}

Starting on the tree implementation, quite a lot of code was given in the task description, and it was simply a matter of filling in the blanks.
Because of this and the fact that the process was the same as described above, we will not go through the code here, but there will be a link at the 
end to where the code can be found.

Finally we had to implement a benchmark to measure the performance of our two implementations. Yet again this was simple a matter of following the 
instructions since the complete code for a benchmark of the list implementation was given. The only thing changed was some of the names for 
clarification. This benchmark was just for the list implementation but works for the tree implementation as well, if one changes the function being 
called.
\section*{Result}
Table \ref{table:bench} shows the result from our benchmark.
\begin{table}[h]
    \centering
    \begin{tabular}{|c c|c|c|c|c|c|}
        \hline
         & \multicolumn{3}{c|}{List} & \multicolumn{3}{|c|}{Tree} \\ \hline
        n & add & lookup & remove & add & lookup & remove  \\ \hline
        16   &     0.56   &     0.23   &     0.38    & 0.22      &0.16      & 0.51    \\
        32   &     0.86   &     0.31   &     0.49  &   0.42    &   0.20   &    0.30  \\ 
        64   &     1.25   &     0.50   &     1.14  &    0.37   &     0.25 &    0.37  \\ 
       128   &     2.51   &     1.06   &     2.33  &     0.50  &    0.29  &    0.55 \\ 
       256   &     5.10   &     1.70   &     4.81  &     0.59  &    0.32  &     0.55 \\ 
       512   &     5.75   &     1.56   &     4.10  &      0.64 &    0.37  &     0.68 \\ 
      1024   &     6.44   &     1.77   &     4.88  &      0.95 &    0.43  &     0.81 \\ 
      2048   &    10.1   &     3.11   &     9.38  &     0.81  &     0.34 &     0.64 \\ 
      4096   &    20.2   &     5.66   &    16.9  &      0.58 &     0.29 &     0.51 \\ 
      8192   &    46.3   &    17.9   &    44.9  &      0.52 &     0.26 &     0.42 \\ 
     16384   &    71.4   &    21.0   &    69.3  &      0.48 &     0.21 &     0.36 \\ 
     32768   &   154   &    39.1   &   141  &           0.67     &  0.25    &  0.47    \\ \hline
    \end{tabular}
\caption{Benchmark of list and tree implementation of a map. \\Values in $\mu s$}
\label{table:bench}
\end{table}

\FloatBarrier
\section*{Discussion}
Link
\href{https://github.com/adrian-jonsson-sjoedin/ID1019-Programming-II/tree/main/Task2_Solution}{GitHub}.




\end{document}
