\documentclass[a4paper,11pt]{article}
\usepackage{graphicx}
\usepackage[utf8]{inputenc}
\usepackage{hyperref}
\usepackage{placeins}
\usepackage[newfloat]{minted}
\usepackage{caption}
\usepackage{amssymb}

\newenvironment{code}{\captionsetup{type=listing}}{}
\SetupFloatingEnvironment{listing}{name=Code Overview}


\hypersetup{
    colorlinks=true,
    linkcolor=blue,
    filecolor=black,      
    urlcolor=blue,
    citecolor=black,
}

\begin{document}

\title{
    \textbf{Task 5 Advent of Code Day 1}
}
\author{Adrian Jonsson Sjödin}
\date{Spring Term 2023}

\maketitle

\section*{Introduction}
In this report we solve the day one puzzle of \href{https://adventofcode.com/2022}{Advent of Code 2022}. The task was to find the 
elf carrying the most calories among a group of elves and return which elf it was and how many calories they where carrying in total.
As input we have the elves inventory where every elf's item's calorie count is listed on a separate line, and each el's inventory is 
separated by a blank line.

\section*{Method}
The solution for the code consists of two functions. The first function, {\tt parse/1}, takes the 
input read from the file as a string and trims it, splits by the blank line to separate by elf 
and then converts each elf's inventory from a list of strings to a list of integers. 

The second function, {\tt solve/1}, is the one that will be called from {\tt iex} to solve the 
problem. It takes the input file path as an argument, reads the content and passes it to the 
{\tt parse/1} function. It then maps each elf's inventory that {\tt parse/1} produce to a list 
containing the total calorie count. It also uses {\tt Enum.with\_index(1)} to add the elf's index 
to the list. The 1 in the argument is because we want to start counting from one and not zero. 
Finally the last step is to return the list with the highest total calories count and the index 
for the elf who had it.


% \begin{minted}{elixir}
%     def deriv({:exp, u, {:num, n}}, v) do
%     {:mul, 
%         {:mul, {:num, n}, {:exp, u, {:num, n - 1}}},
%         deriv(u, v)
%     }
%   end
% \end{minted}



\section*{Result}
The code is tested with the text file seen in code overview \ref{code:inventory} and 
produces the output seen in code overview \ref{code:result}. It means that the elf with 
the highest calorie count is elf number 1 with a calorie count if 26,000 calories in its 
inventory. Doing a manual check against the file we see that this is indeed correct.\\
\begin{code}
\captionof{listing}{inventory.txt}
\label{code:inventory}
\begin{minted}{bash}
                            1000
                            2000
                            3000
                            20000

                            4000

                            5000
                            6000

                            7000
                            8000
                            9000

                            10000
\end{minted}
\end{code}
\begin{code}
\captionof{listing}{Result of running AOC.solve("inventory.txt")}
\label{code:result}
\begin{minted}{bash}
                        {[26000], 1}
\end{minted}
\end{code}

\section*{Discussion}
The challenge with this task was to understand how to correctly use the pipe operator 
{\tt |>} and the {\tt \&} symbol which is used to create an anonymous function capture.
This was needed to be able to pass the result from one operation to the next. It did 
however help by thinking of them as similar to how you can chain functions calls together 
in JavaScript, and how you use the pipe operator in bash.

The solution itself is quite straightforward and unlike previous task did not require any 
recursion. 

One drawback of this solution is that it assumes that the file we read from 
have all data in the correct format. We do not check for any potential errors or do any 
form of error handling. However this was not part of the task so I think this solution is 
justified.

Another thing that could be improved is the reporting of more relevant information, like
the total number of elves or the size of each elf's inventory.

The code in its entirety can be found here:
\href{https://github.com/adrian-jonsson-sjoedin/ID1019-Programming-II/tree/main/Task5_Solution}{GitHub}.




\end{document}
